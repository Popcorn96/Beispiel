\documentclass[a4paper,12pt,oneside,onecolum,final,openany]{report} 

%---- Allgemeine Layout Einstellungen ------------------------------------------

% Für Kopf und Fußzeilen, siehe auch KOMA-Skript Doku
\usepackage[komastyle]{scrpage2}
\pagestyle{plain}
\setheadsepline{0.5pt}[\color{black}]
\automark[section]{chapter}


%Einstellungen für Figuren- und Tabellenbeschriftungen
\usepackage[utf8x]{inputenc} 
\usepackage[T1]{fontenc} 
\usepackage[ngerman]{babel} 
\usepackage{amsfonts} 
\usepackage{amsmath} 
\usepackage{tabularx}
\usepackage{float}

%---- Weitere Pakete -----------------------------------------------------------
% Die Pakete sind alle in der TeX Live Distribution enthalten. Wichtige Adressen
% www.ctan.org, www.dante.de

% Sprachunterstützung
\usepackage[ngerman]{babel}

% Benutzung von Umlauten direkt im Text
% entweder "latin1" oder "utf8"
%\usepackage[utf8]{inputenc}

% Pakete mit Mathesymbolen und zur Beseitigung von Schwächen der Mathe-Umgebung
%\usepackage{latexsym,exscale,stmaryrd,amssymb,amsmath}


\usepackage[nointegrals]{wasysym}
\usepackage{eurosym}

% Anderes Literaturverzeichnisformat
%\usepackage[square,sort&compress]{natbib}
\usepackage{hyperref}
% Für Farbe
\usepackage{color}
\usepackage{graphicx}
\usepackage{wrapfig}
\usepackage{subfigure}

% Caption neben Abbildung
\usepackage{sidecap}


% Befehl für "Entspricht"-Zeichen
\newcommand{\corresponds}{\ensuremath{\mathrel{\widehat{=}}}}
% Befehl für Errorfunction
\newcommand{\erf}[1]{\text{ erf}\ensuremath{\left( #1 \right)}}


%Fußnoten zwingend auf diese Seite setzen
\interfootnotelinepenalty=1000

%Für chemische Formeln (von www.dante.de)
%% Anpassung an LaTeX(2e) von Bernd Raichle
\makeatletter
\DeclareRobustCommand{\chemical}[1]{%
  {\(\m@th
   \edef\resetfontdimens{\noexpand\)%
       \fontdimen16\textfont2=\the\fontdimen16\textfont2
       \fontdimen17\textfont2=\the\fontdimen17\textfont2\relax}%
   \fontdimen16\textfont2=2.7pt \fontdimen17\textfont2=2.7pt
   \mathrm{#1}%
   \resetfontdimens}}
\makeatother
\usepackage{textcomp}
\usepackage{upgreek}
%\begin{document}
%$\upmu$
%\end{document}
%Honecker-Kasten mit $$\shadowbox{$xxxx$}$$
\usepackage{fancybox}

%SI-Package
\usepackage{siunitx}

%keine Einrückung, wenn Latex doppelte Leerzeile
\parindent0pt

%Bibliography \bibliography{literatur} und \cite{gerthsen}
%\usepackage{cite}
\usepackage{babelbib}
\selectbiblanguage{ngerman}

\usepackage{siunitx}
%\begin{document}
 % \SI{1.55}{\micro\metre}
\sisetup{math-micro=\text{µ},text-micro=µ}
\usepackage{amsmath}
\begin{document}

\begin{titlepage}
\centering
\textsc{\Large Physikalisch- Chemisches Grundpraktikum\\[1.5ex] Universität Göttingen}

\vspace*{0.5cm}

\rule{\textwidth}{1pt}\\[0.5cm]
{\huge \bfseries
  Versuch 12: \\[1.5ex]
  Joule-Thomson-Effekt}\\[0.5cm]
\rule{\textwidth}{1pt}

\vspace*{0.5cm}


\begin{Large}
\begin{tabular}{ll}
Durchführende: &  Isaac Maksso, Julia Stachowiak\\
Assistentin: & Katharina Meyer \\
 Versuchsdatum: & 27.10.2016\\
 Datum der ersten Abgabe: & 03.11.2017\\
\end{tabular}
\end{Large}

\vspace*{0.5cm}

 Messergebnisse:\\
\begin{table}[h]
\centering 
 \begin{tabular}{c|c|c|c|c}
 Temperatur/~°C&$\text{N}_2^{\text{Exp.}}$/~$\frac{\text{K}}{\text{bar}}$&$\text{CO}_2^{\text{Exp.}}$/~$\frac{\text{K}}{\text{bar}}$&$\text{N}_2^{\text{Th.}}$/~$\frac{\text{K}}{\text{bar}}$&$\text{CO}_2^{\text{Th.}}$/~$\frac{\text{K}}{\text{bar}}$\\
 \hline
 0 &1.20±0.038&0.181±0.022&0.031&0.071\\
 \hline
 20 &1.01±0.060&0.175±0.024&0.027&0.064\\
\hline 
50 &0.710±0.047&0.120±0.013&0.021&0.055\\
 \end{tabular}
\end{table}
 Literaturwerte:\\
\begin{table}[h]
\centering
\begin{tabular}{c|c|c}
  &$\text{N}_2^{\text{Lit.}}$/~$\frac{\text{K}}{\text{bar}}$&$\text{CO}_2^{\text{Lit.}}$/~$\frac{\text{K}}{\text{bar}}$\\
\hline
 0~°C &$0.26^{5}$ & $1.31^{5}$\\
\hline
20~°C&$0.25^{4}$& $1.12^{4}$\\
\hline
50~°C&$0.19^{5}$ &$0.91^{5}$\\
\end{tabular}
\end{table} 
\end{titlepage}
\tableofcontents
\chapter{Experimentelles}
\section{Experimenteller Aufbau}
\begin{figure} [h!]
\begin{center}
\includegraphics[scale=1.5]{VersuchsaufbauJT.png} \end{center}
\caption{Versuchsaufbau.}
\end{figure}
a) Gasanschluss\\
b)Thermometer\\
c)Gaseinlass\\
d)Drosselstelle\\
e)Differenzdruckmanometer\\
f)Wärmebad\\
g)Wärmetauscher\\
h)$\Delta$T-Messstelle\\
\section{Durchführung}
Es wurde eine Druckdifferenz von 0.1 atm eingestellt. Das Temperaturdifferenz-Messgerät wurde auf 0 gestellt. Es wurde in 0.2 atm Schritten bis zur einer Druckdifferenz von 1.7 atm erhöht und die Temperaturdifferenz nach 10 Sekunden Wartezeit notiert. Bei 1.7 atm wurde um 0.5 atm erhöht und die Temperatur gemessen. Von 1.75 atm wurde in 0.2 atm Schritten die Druckdifferenz erniedrigt und die Temperaturdifferenz notiert. Dieser Messvorgang wurde mit Stickstoff und Kohlenstoffdioxidgas bei Zimmertemperatur, dann bei 0~°C und zu letzt bei 50~°C durchgeführt.
\chapter{Auswertung}

%Darstellung und Auswertung der Messergebnisse
%Fehlerrechnung und Fehlerdiskussion (statistische und systematische fehler)
%Abschätzung systematischer Fehler
%Endergebnis mit Fehlerangabe


\section{Berechnung des mitteleren Joule-Thomson-Koeffizienten}
Der mittlere Joule-Thompson-Koeffizient entspricht der Steigung bei einer Auftragung von $\Delta$ T gegen $\Delta$ p. Hierbei wird der Differentialquotient durch einen Differenzenquotienten ersetzt. Der Fehler ergibt sich als der Fehler der Auftragung.\\

\begin{equation}
T_1 - T_2 = <\mu_{\text{JT}}> \cdot (p_1 - p_2)
\end{equation}

Die sich ergebenden Steigungen sind in nachfolgender Tabelle aufgelistet. Die Fehler wurden als Standardfehler der Auftragung berechnet. Die einzelnen Auftragungen sind im Anhang zu finden.\\

\begin{table} [h]
\centering
\caption{Experimentell bestimmt Joul-Thomson-Koeffizienten $\mu_{\text{JT}}^{\text{Exp.}}$}
\begin{tabular} {l | c|  c | c}
	 &  T [K] & $<\mu_{\text{JT}}^{\text{Exp.}}>$/~$\frac{\mathrm{K}}{\mathrm{bar}}$ & $\Delta <\mu_{\text{JT}}^{\text{Exp.}}>$/~$\frac{\mathrm{K}}{\mathrm{bar}}$\\
	 \hline
	  $CO_\mathrm{2}$ & 273.25 & 1.20 & 0.038 \\
	   & 295.95 & 1.01 & 0.060\\
	  & 323.95 & 0.710 & 0.047\\
	\hline
	$N_\mathrm{2}$ & 273.25 & 0.181  &  0.022\\
	& 295.95 & 0.175 & 0.024\\
	& 323.95& 0.120& 0.013\\
\end{tabular}
\end{table}
\section{Herleitung $c_\mathrm{v}^\mathrm{ vib}$}

Zur Berechnung von $\mu_{JT}$ wird neben tabellierten Größen vor allem die molare Wärmekapazität bei konstantem Druck $c_\mathrm{p}$ benötigt. Diese wiederum ergibt sich aus der molaren Wärmekapazität bei konstantem Volumen $c_\mathrm{v}$:

\begin{equation}
c_\mathrm{p} = c_\mathrm{v} +R
\end{equation}

Da Wärme in Form von Schwingungsenergie(Vibration), Rotationsenergie und Bewegungsenergie(Translation) von einzelnen Molekülen aufgenommen und somit gespeichert werden kann, liefern diese je ihren Beitrag zur molaren Wärmekapazität(die Energieaufnahme in Form von Elektronenanregung spielt in diesem Fall keine Rolle und wird daher vernachlässigt):

\begin{equation}
c_\mathrm{v} = c_\mathrm{v}^\mathrm{ vib} + c_\mathrm{v}^\mathrm{ rot} + c_\mathrm{v}^\mathrm{trans}  
\end{equation}

Die einzelnen Beiträge ergeben sich je aus den Freiheitsgraden für Moleküle. Die sind je nach Atomzahl und Geometrie (linear/gewinkelt) unterschiedlich. $N_2$ und $CO_2$ sind beide linear und haben daher ähnliche Werte:\\


\begin{table} [h]
\centering
\caption{$\text{c}_v$-Wert der Translations, Rotation und Schwingung von $\text{N}_2$ und $\text{CO}_2$.}
\begin{tabular} {l | c|  c | c}
	 & $c_\mathrm{v}^\mathrm{ vib}$  & $c_\mathrm{v}^\mathrm{ rot}$ & $c_\mathrm{v}^\mathrm{trans}$\\
	 \hline
	 $N_2$ & $k_B$T & $k_B$T &$\frac{3}{2}k_B$T\\
	  $CO_2$ & 4$k_B$T & $k_B$T &$\frac{3}{2}k_B$T\\
\end{tabular}
\end{table}

Da die Schwingung erst bei sehr hohen Temperaturen vollständig angeregt ist, muss folgende Formel verwedet werden, die hier hergeleitet wird: \\


\begin{equation}
c_\mathrm{v}^\mathrm{ vib} = R\cdot(hv\beta)^2 \frac{\mathrm{exp}(hv\beta)}{\mathrm{exp}(hv\beta -1)^2}
\end{equation}


mit $\beta=\frac{1}{k_{B} \cdot T}$.\\

Grundlage bildet die Vibrationszustandssumme des Systems:\\

\begin{equation}
Z_{vib}= \left[\frac{\mathrm{exp}(-\frac{hv}{2k_BT})}{1-\mathrm{exp}(-\frac{hv}{k_BT})}\right]^N
\end{equation}

Bei adiabatischer Prozessführung ist die Wärmezufuhr/-abgabe gleich der Änderung der inneren Energie. Die Vibrationszustandssumme steht mit letzterer folgendermaßen im Zusammenhang:\\

\begin{equation}
U_{vib}= -k_BT^2 \frac{\partial lnZ_{vib}}{\partial T} = N\frac{hv}{2} +\frac{Nhv}{exp\left(\frac{hv}{k_BT}\right)-1}= U_\mathrm{m}^\mathrm{vib} = R \Theta \left[ \frac{1}{2} + \frac{1}{\mathrm{exp}(\frac{\Theta_\mathrm{vib}}{T} -1)}\right]
\end{equation}

Mit\\

\begin{equation}
\Theta_\mathrm{vib} = \frac{hv}{k_\mathrm{B}}
\end{equation}


Die Ableitung der molaren inneren Energie ergibt die molare Wärmekapazität:\\


\begin{equation}
c_\mathrm{v} = \frac{\partial U_\mathrm{m}}{\partial T}\bigg \vert_\mathrm{V}
\end{equation}

Analog gilt:

\begin{equation}
c_\mathrm{v}^\mathrm{vib} = \frac{\delta U_\mathrm{m}^\mathrm{vib}}{\delta T}\bigg \vert_\mathrm{V}
\end{equation}

Das ganze abgeleitet ergibt wiederum $c_v^{vib}$ als herzuleitende Formel.\\


\section{Berechnung von $\mu_{JT}$ aus der Virialgleichung}

Der Unterschied eines realen Gases zum idealen Gas wird mittels Virialentwicklung beschrieben, wobei hier ein Abbruch nach dem 2. Virialkoeffizienten stattfindet.\\

\begin{equation}
pV_\mathrm{m} = RT + B(T) \cdot p
\end{equation}

Wie in obiger Gleichung gut zu sehen ist, bildet der 2. Virialkoeffizient einen Korrekturterm für den Druck im Bezug zum idealen Gasgesetz, der linear vom Druck abhängig ist.
Der Joule-Thomson-Koeffizient kann in Abhängigkeit des 2. Virialkoeffizienten folgendermaßen ausgedrückt werden:

\begin{equation}
\mu_{JT} = \frac{1}{c_\mathrm{p}}\left( T \frac{\delta B}{\delta T} \bigg \vert_\mathrm{p} - B\right)
\end{equation}

Hier wird mit der reduzierten Temperatur gerechnet. dh.: 

\begin{equation}
T^* = \frac{T}{\varepsilon}
\end{equation}

$\varepsilon$ ist hierbei eine Stoffkonstante der Einheit Kelvin, die für $N_\mathrm{2}$ und $CO_\mathrm{2}$ unterschiedliche Werte annimmt. Somit ist die redizierte Temperatur eine dimensionslose Größe, mit der sich leichter rechnen lässt.\\ 
Somit wird $\mu_{JT}$ letzendlich folgendermaßen berechnet:\\

\begin{equation}
\mu_{JT} = \frac{1}{c_\mathrm{p}}\left( T^* \frac{\partial B^*(T^*)}{\partial T^*} \bigg \vert_\mathrm{p} - B^*(T^*)\right)
\end{equation}

Die Werte für $B^*(T^*)$ und $\mathrm{T^*}\frac{\mathrm{dB^*(T^*)}}{\mathrm{dT^*}}$ pro $T^*$ wurden einer Tabelle am Ende des Praktikumsskriptes entnommen.\protect\footnote{Hirschfelder, Curtiss, Bird, New Yorck 1954}

Folgende Größen ergeben sich:\\

\begin{table} [h]
\centering
\caption{Theoretisch bestimmte Joul-Thomson-Koeffizienten $\mu_{\text{JT}}^{\text{Th.}}$ für $\text{N}_2$  und $\text{CO}_2$.}
\begin{tabular} {l | c|  c | c}
	 &  T [K] &  $c_\mathrm{p}$ [$\frac{\mathrm{J}}{\mathrm{K}\cdot \mathrm{mol}}$] & $ <\mu_{\text{JT}}^{\text{Th.}}>$ [$\frac{\mathrm{K}}{\mathrm{bar}}$] \\
	 \hline
	  $\text{N}_\mathrm{2}$ & 273.25 & 37.4 &  0.031\\
	   & 295.95 & 37.4 & 0.027\\
	  & 323.95 & 37.4 & 0.021 \\
	\hline
	$\text{CO}_2$ & 273.25& 54.0&0.071  \\
	& 295.95 & 54.0& 0.064\\
	& 323.95& 54.0& 0.055\\
\end{tabular}
\end{table}
Da $CO_2$ drei unterschiedliche Schwingungsfrequenzen besitzt, wurde $c_\mathrm{v}^\mathrm{ vib}$ je berechnet und die Werte summiert.\\

%$\frac{\partial B^*}{\partial T^*}= -0,241$ Außerdem gilt die Beziehung:\\

%\begin{equation}
%B(T) = b_\mathrm{0}B^*(T^*) = b_0(0,904 -0,241T^*)
%\end{equation}

%Mit $b_0= \frac{2}{3}\pi N_\mathrm{L} \sigma^3$ und  lässt sich somit der Joule- Thompson- Koeffizient aus der Wärmekapazität bei konstantem Druck errechnen:\\

%\begin{equation}
%\mu_{JT} = \frac{1}{c_\mathrm{p}} \left(T\left(-0,241b_0\right) - b_0\left(0,904-0,241T^*\right)\right)
%\end{equation}

%Letzendlich wurden die $\mu_{JT}$-Koeffizienten mit folgender Formel ermittelt:\\

%\begin{equation}
%\frac{1}{\mathrm{c_p}}\left(T^*\frac{dB^*(T^*)}{dT^*} -B^*(T^*)\right)
%\end{equation}

\section{Fehlerrechnung}
Die Ungenauigkeit der Auftragung wurde mittels der Funktion\textit{Linear Fit} des Programms \textit{Origin 8.5G} bestimmt. Es wurde die Standerabweichung genutzt. \\
Die Ungenauigkeit der $\mu_{\text{JT}}^{\text{Th.}}$ wurden nicht bestimmt, da dieser Wert aus Literaturwerten bestimmt wurde.
%\subsection{Vergleich mit den Literaturwerten}
%Für die Molmasse der Substanz im Cyclohexan ergeben sich folgende Werte:\\
%$M_A = (13 \cdot 10 \pm 3 \cdot 10){~} \mathrm{g{~}mol^{-1}}$\\\\
%$M_B = (12 \cdot 10 \pm 2 \cdot 10){~} \mathrm{g{~}mol^{-1}}$\\\\
%Dabei handelt es sich um Campher, dieses hat eine Molmasse von $ 152,23{~} \mathrm{g{~}mol^{-1}}$ \footnote{https://de.wikipedia.org/wiki/Campher abgerufen am \textbf{13.12.2015}} Beide Werte weichen um $15-21\%$ nach unten hin ab, dies legt einen systematischen Fehler nahe, wobei der Wert 1 noch im Fehlerintervall liegt.\\\\
%Für die Molmasse der im Wasser gelösten Substanz ergibt sich:\\\\
%$M_A = (6 \cdot 10 \pm 3 \cdot 10){~} \mathrm{g{~}mol^{-1}}$\\\\
%$M_B = (5\cdot 10 \pm 3\cdot 10 ){~} \mathrm{g{~}mol^{-1}}$\\\\
%Es handelt sich bei der im Wasser gelösten Substanz um Kaliumchlorid, dieses hat eine Molmasse von $74,55{~} \mathrm{g{~}mol^{-1}}$ {~}\footnote{https://de.wikipedia.org/wiki/Kaliumchlorid abgerufen am \textbf{09.12.2015}} Die starke Abweichung von $20-30\%$ der Werte lässt sich nicht mithilfe der Fehlerfortpflanzung erklären, daher muss es sich um systematische Fehler handeln. 
\chapter{Diskussion}
Die Auswertungsergebnisse und die Literaturwerte sind in Tabelle 4 dargestellt.:
\begin{table}[h]
\centering
\caption{Literaturwerte des Joul-Thomson-Koeffizienten für $\text{N}_2$ und $\text{CO}_2$ bei verschiedenen Temperaturen.}
\begin{scriptsize}
\begin{tabular}{c|c|c|c|c|c|c}
Temperatur/~°C&$\text{N}_2^{\text{Exp.}}$/~$\frac{\text{K}}{\text{bar}}$&$\text{CO}_2^{\text{Exp.}}$/~$\frac{\text{K}}{\text{bar}}$&$\text{N}_2^{\text{Th.}}$/~$\frac{\text{K}}{\text{bar}}$&$\text{CO}_2^{\text{Th.}}$/~$\frac{\text{K}}{\text{bar}}$&$\text{N}_2^{\text{Lit.}}$/~$\frac{\text{K}}{\text{bar}}$&$\text{CO}_2^{\text{Lit.}}$/~$\frac{\text{K}}{\text{bar}}$\\
\hline
0 &0.181±0.022&1.20±0.038&0.031&0.071&$0.26^{5}$ & $1.31^{5}$\\
\hline
20 &0.175±0.024&1.01±0.060&0.027&0.064&$0.25^{4}$& $1.12^{4}$\\
\hline
50&0.120±0.013 &0.710±0.047&0.021&0.055&$0.19^{5}$ &$0.91^{5}$\\
\end{tabular}
\end{scriptsize}
\end{table}
\noindent
Keiner der experimentell bestimmten Joul-Thomson-Koeffizienten ist dem entsprechenden Literaturwert nahe. Die bestimmten Koeffizienten sind kleiner als die Literaturwerte. Ursache hierfür könnte die Messmethode sein, die die Bestimmung der Temperaturdifferenz durch schwankende Werte und abnehmende Druckdifferenz nicht exakt ist.\\
Die theoretisch bestimmten Joul-Thomson-Koeffizienten unterscheiden sich vom Literaturwert um eine Zehnerpotenz. Die fehlerbehaftete Messgröße ist die Temperatur. Die aus diesen Messwerten tabellisierten Wert für T* und B(T*) sind in großen Schritten aufgezeichnet, sodass die Ungenauigkeit der Temperatur auf diese Größe gering ist. Des Weiteren ist der Anteil der Ungenauigkeit auf den Wert der Wärmekapazität gering, wodurch die Abweichung der theoretisch bestimmten Koeffizienten nicht erklärt werden kann.\\
\chapter{Anhang}
\begin{center}
\begin{figure}[h]
\includegraphics[width=13.5cm]{N2bei0.png}
\caption{$\text{N}_2$ bei 0~°C.}
\end{figure}
\end{center}
\begin{center}
\begin{figure}[h]
\includegraphics[width=13.5cm]{N2bei22,7.png}
\caption{$\text{N}_2$ bei 20~°C.}
\end{figure}
\end{center}
\begin{center}
\begin{figure}[h]
\includegraphics[width=13.5cm]{N2bei50.png}
\caption{$\text{N}_2$ bei 50~°C.}
\end{figure}
\end{center}
\begin{center}
\begin{figure}[h]
\includegraphics[width=13.5cm]{CO20Grad.png}
\caption{$\text{CO}_2$ bei 0~°C.}
\end{figure}
\end{center}
\begin{center}
\begin{figure}[h]
\includegraphics[width=13.5cm]{CO2bei22,8.png}
\caption{$\text{CO}_2$ bei 20~°C.}
\end{figure}
\end{center}
\begin{center}
\begin{figure}[h]
\includegraphics[width=13.5cm]{CO2bei50,8.png}
\caption{$\text{CO}_2$ bei 50~°C.}
\end{figure}
\end{center}
\newpage


\chapter{Literaturverzeichnis}
1\quad Eckhold, Götz: \emph{Praktikum I zur Physikalischen Chemie}, Institut für Physikalische Chemie, Uni Göttingen, \textbf{2014}.

\vspace{0,5 cm}

2 \quad Eckhold, Götz: \emph{Statistische Thermodynamik}, Institut für Physikalische Chemie, Uni Göttingen, \textbf{2012}.

\vspace{0,5cm}

3 \quad Eckhold, Götz: \emph{Chemisches Gleichgewicht}, Institut für Physikalische Chemie, Uni Göttingen, \textbf{2015}.\\

\vspace{0,5cm}

4 \quad Atkins, P.W.: \emph{Physikalische Chemie}, Wiley-VCH, Weinheim, \textbf{2006}.\\

\vspace{0,5cm}

5 \quad Zemansky: \emph{Heat and Thermodynamics},Mc Graw-Hill, New York, \textbf{1990}.\\

\end{document}
